\documentclass[12pt, a4paper]{article}
\usepackage[brazil]{babel}
\usepackage[utf8]{inputenc}
\usepackage[alf]{abntcite}
\usepackage{caption} 
\usepackage{setspace}

\usepackage[left=3.00cm, right=2.00cm, top=3.00cm, bottom=2.00cm]{geometry}


\makeatother

\begin{document}
\onehalfspace

\title{Consultas Distribuídas com Banco de Dados Relacional: um modelo de escalabilidade horizontal utilizando sharding}
\author{Fabiano Sardenberg Kuss}
%\orientador{}
%\instituicao{SERPRO}
%\local{Curitiba - PR}


%\begin{resumo}

%\end{resumo}


\section{Introdução}

Uma modelo de base de dados distribuídas utilizando sistemas de gerenciamento de bancos de dados, SGBDs, relacionais é uma
ferramenta que provê escalabilidade horizontal em soluções utilizando ferramentas que permitam que agregar 
as funcionalidades do modelo relacional em uma arquitetura onde a persistência dos dados não é feita em
uma mesma máquina física ou virtual. Esta abordagem de manutenção dos dados em um ambiente distribuído 
permite que equipamentos ampliar a capacidade de manipulação e manutenção de dados que compreendem grande
consumo de recursos computacionais seja realizado por equipamentos com reduzida capacidade de processamento.
A ampliação do número de usuários e da quantidade de dados mantidos pela infraestrutura corporativa pode
ser redimensionada de acordo com as demandas que venham a surgir ao longo do tempo sem que haja a necessidade
de dimensionamento prévio de processamento, armazenamento e memória.

Um dos grandes desafios na utilização de SGBDs relacionais distribuídos é a própria concepção
do modelo que depende do atendimento de premissas concebidas para garantir a integridade dos dados e operações
que implicam em processos pouco adequados a ambientes distribuídos do ponto de vista do tempo de execução
das tarefas de persistência das informações. A necessidade de garantia de transações utilizando um único 
ponto onde são mantidas os registros implicam em confirmações e bloqueios de acesso a determinados dados
para atender a aspectos de garantia de unicidade das informações. Este modelo de persistência de dados
foi concebida em um momento em que a computação ainda possuía um grande enfoque em modelos centralizados
de processamento de dados e atendeu muito bem a maioria das demandas até o início da década de 2000 \cite{kossmann2000state} mas
com a proposta de maior escalabilidade de recursos de hardware para atendimento pontual das necessidades
de maneira que não exista superdimensionamento de infraestrutura em decorrência de expectativas que podem
nem sempre se realizar levantou um questionamento contundente da adequação do modelo centralizado as novas
demandas das ferramentas de Tecnologia da Informação e Comunicação, TICs.

Neste panorama de questionamento de modelos de persistência de dados em ambientes distribuídos surgiram 
propostas e ferramentas que tem sido defendidas como mais adequadas que os tradicionais SGBDs relacionas
denominados com bancos de dados noSQL, \textit{not struturad query language}, em tradução livre não SQL, que foram
rapidamente aceitos como solução para organizações dotadas de grande demandas de manutenção de informações
tais como sistemas de busca na Internet e ambientes de redes sociais \cite{cattell2011scalable}. Existem notadas vantagens do ponto
de vista de utilização destas ferramentas em ambientes distribuídos pois várias delas foram concebidas
justamente para prover performance nestas arquiteturas tecnológicas e são criadas para prover melhor
desempenho em situação específicas. Estas ferramentas, no entanto, apresentam um modelo de uso diferenciado
para cada solução não existindo, como no caso dos bancos de dados relacionais SQL, uma forma relativamente
semelhante de manipulação de informações, motivo este que impõe grande dependência da aplicação a
ferramenta escolhida para a persistência dos dados.

Muitas das soluções de software que tem grande dependência da manutenção e recuperação de informações
em banco de dados foram desenvolvidas utilizando o modelo relacional. Vários dos gestores de projetos 
e desenvolvedores de aplicação tiveram, em sua formação e prática profissional, experiências apanas
com ferramentas SQL e, dado ao relativo sucesso na maioria das implementações utilizando estas ferramentas,
sequer vislumbram a possibilidade de abordagem de outras opções para soluções que utilizem 
bancos de dados. A mudança desta visão de utilização de repositórios sem bancos relacionais
implica em uma mudança significativa na abordagem de desenvolvimento de sistemas e pode representar
necessidades de reimplementação que demanda custos impraticáveis, a curto prazo, para a maioria das
organizações \cite{Tran_KLF_11}.

A avaliação do desempenho de modelos distribuídos utilizando banco de dados relacionais é um importante
fator na decisão na escolha de implementar esta abordagem para a melhoria da qualidade de sistemas
que apresentem a necessidade de prover crescimento de infraestrutura ou distribuição das informações 
em ambientes geograficamente distantes. Consolidar informações relativas ao desempenho da solução 
utilizando distribuição dos dados permite a tomada de decisões de implementação que permitam a 
geração de resultados mais consistentes e adequados a necessidade de escalabilidade permitindo tanto
a definição da arquitetura mais adequada ao projeto quanto avaliar os impactos no desenvolvimento
da solução. Em casos em que já existe uma implementação baseada em modelos centralizados testes de 
performance permitem avaliar os impactos de alterações na implementação dos sistemas e definições
de estratégias que promovam menor impacto na adequação da solução podendo inclusive apontar a inviabilidade
de determinados projetos adequarem-se ao modelo distribuído \cite{lin2011tenzing}.

Durante o processo de revisão de literatura foi identificado um certo abandono de estudos que tratem
do uso de bancos de dados relacionais em ambientes distribuídos. A nova abordagem apresentada neste
trabalho propõe que muitos dos sistemas de informação que manipulam informações utilizando SGBDs podem
ser mantidos sem a necessidade de equipamentos robustos e ma

O objetivo geral deste trabalho é avaliar a possibilidade de uso de bancos de dados relacionais em ambientes distribuídos a 
partir do uso de consultas distrubuídas. Decorre desta avaliação a necessidade de apresentar instruções SQL e 
sequências de execuções que possam ser realizadas em bancos de dados relacionais distribuídos; propor estratégias 
de consultas que façam a intersecção de dados entre tabelas, consultas \textit{join} e apresentar
um modelo de implementação com mínimo impacto no desenvolvimento ou migração de aplicações.



\section{Trabalhos correlacionados}

\section{Metodologia}

Este trabalho é de caráter exploratório e pretende apresentar um modelo de implementação de consultas
utilizando bancos de dados relacionais em ambiente distribuído. Para a realização dos experimentos
foi utilizado o SGBD PostgreSQL dada sua popularidade, possibilidade de acesso ao código fonte, e 
familiaridade dos pesquisadores com a ferramenta. As instruções SQL utilizadas neste trabalho foram
baseadas nas definições apresentadas por Snodgrass \cite{snodgrass1994tsql2}.

Inicialmente serão avaliados sistemas de informação desenvolvidos ou mantidos pelo Serviço Federal de
Processamento de Dados, Serpro, 



\section{Analise dos dados}

As instruções utilizadas nas consultas de teste visam cobrir apenas conjuntos de instruções SQL mais simples
mas os resultados obtidos permitem suposições de tempo de execução de consultas mais complexas utilizando-se
modelos que representem cada uma das instruções separadamente.

%\begin{table}[ht]
\begin{tabular}{l l}
\hline
Mnemônico & Instrução \\
\hline
TSA & SELECT * FROM test \\
TSC & SELECT count(test\_id) FROM test \\
TSW & SELECT * FROM test WHERE test\_id = 20\\
TSL & SELECT * FROM test WHERE value\_string like '1 \%'\\
CSA & SELECT * FROM test\_child1\\
CSJ & SELECT * FROM test\_child1 t1, test t where t.parent\_id = t1.id\\
BSA & SELECT * FROM test\_child\_blob\\
MSA & SELECT * FROM test\_child\_multi\_col\\
\hline
\end{tabular}
%\caption{Instruções SQL dos testes}
%\label{table:instrucoes}
%\end{table}

Os servidores de banco de dados utilizados para a realização dos testes foram mensurados em relação a
sua capacidade de processamento de instruções SQL utilizando a instrução identificada pelo mnemônico
TSC e a velocidade de rede em relação ao tempo médio de execução de cem instruções ping. A arquitetura
do equipamento é irrelevante para o estudo proposto e foram considerados apenas o tempo de rede independentemente
da infraestrutura para a geração dos resultados que serão aplicados como base das validações.

\begin{tabular}{l l l}
\hline
Mnemônico & Ping & TSC \\
\hline
M1 & 0.8 & 0.12 \\
M8 & 0.8 & 0.12 \\
\hline
\end{tabular}


Para avaliar o comportamento da execução considerando o crescimento do número de registros e relação do
número de campos retornados nas consultas foram obtidos os seguintes resultados médios da execução.

\begin{tabular}{l l l l l l l l l l}
\hline
Mnemônico & Registros & TL &TLP &TLM & TLMd & TRP & TRM & TRMd & TA \\
\hline
TSA & 10K & 0.1 & 0.1 & 0.00 & 0.1 & 0.1  & 0.00 & 0.1 & 0.1 \\
MSA & 100K & 0.3 & 0.2 & 0.00 & 0.1 & 0.1  & 0.00 & 0.1 & 0.1 \\

\hline
\end{tabular}



\section{Resultados}


Utilizar uma arquitetura de servidores heterogênea implica em redução do tempo de execução de todo
sistema desenvolvido para consultas compartilhadas pois a dependência de respostas de todos os
nós que compreendem o sistema implica que o tempo de resposta da consulta distribuída é o tempo
de execução do nó mais lento que atende a requisição. O critério de utilização de servidores 
com menor capacidade de processamento pode ser uma solução viável quando estes mantém quantidades
de registros significativamente menores que os equipamentos mais rápidos. Para consultas que
retornam poucos dados a velocidade da rede tem pouco impacto no tempo de execução total de
consultas mas representa um impacto maior quando ocorrem muitos resultados no retorno da mesma.

\section{Considerações Finais e Trabalhos Futuros}

\bibliography{referencias}{}



\end{document}


